\documentclass[12pt]{report}
\usepackage{amssymb,amsmath,color,graphicx}
\topmargin = -.5in
\textheight = 9in
\parindent=0pt
\parskip=12pt

\begin{document}
\centerline{Matlab Basics: Day 1 }
\centerline{\it EDGE 2019}
\vskip .2in

You should have the ``Command Window" open in front of you.  
\begin{enumerate}
\item {\bf How to declare and manipulate vectors and matrices.}

To declare a variable to equal a number just set it equal to that number. So to set the variable ``$x$" equal to 47 type:
\begin{verbatim}
x = 47
\end{verbatim}
followed by ``enter".  You should see:
\begin{verbatim}
x =

    47
\end{verbatim}
which means that Matlab thinks that $x$ is now equal to 47.  (From now on, it is implied that you type ``enter" after a given command.)  If you type a semi-colon after the command before typing ``enter", you won't see any output, but $x$ will still equal 47.  Try it:
\begin{verbatim}
x=4747;
\end{verbatim}
Now see what Matlab thinks $x$ is:
\begin{verbatim}
x
\end{verbatim}
and you should see
\begin{verbatim}

x =

        4747
  \end{verbatim}
  To declare a matrix, put the numbers in square brackets, and separate rows by semi-colons. You may separate entries by commas for legibility, but that isn't necessary.
  \begin{verbatim}
 A = [1 2 3; 4 5 6; 7 8 9];  
  \end{verbatim}
  You should see that $A$ is a $3 \times 3$ matrix. Type
  \begin{verbatim}
  A
\end{verbatim}
to get
\begin{verbatim}
A =

     1     2     3
     4     5     6
     7     8     9
 \end{verbatim}
 Multiplication is with an asterisk:
 \begin{verbatim}
  47 * 47

ans =

        2209
 \end{verbatim}
 or to multiply matrices:
 \begin{verbatim}
 A * A

ans =

    30    36    42
    66    81    96
   102   126   150
\end{verbatim}
Suppose you have two vectors:
\begin{verbatim}
x = [1 2 3];
y = [4 5 6];
\end{verbatim}
and you want to multiply each element of $x$ by the corresponding element in $y$. You can do ``component-by-component" multiplication
using the `.*' command:
\begin{verbatim}
 x.*y

ans =

     4    10    18
 \end{verbatim}
 The same goes for division or exponentiation, which is done with a carat \quad $\hat{}$\quad :
 \begin{verbatim}
 y.^x

ans =

     4    25   216
\end{verbatim}
You can also invert matrices, find their determinants and eigenvalues:
\begin{verbatim}
>> A = [4 2; 1 3]

A =

     4     2
     1     3

>> inv(A)

ans =

    0.3000   -0.2000
   -0.1000    0.4000

>> det(A)

ans =

    10

>> eig(A)

ans =

     5
     2
\end{verbatim}
\pagebreak

To get the $n\times n$ identity matrix, use eye(n):
\begin{verbatim}
>>eye(3)

ans =

    1    0    0
    0    1    0
    0    0    1
 
\end{verbatim}
Similarly, zeroes(n) and ones(n) will give you, respectively, an $n\times n$ matrix of zeroes and ones. 

\end{enumerate}
\end{document}