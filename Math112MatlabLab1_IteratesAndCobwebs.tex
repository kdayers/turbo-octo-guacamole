\documentclass[12pt]{report}
\usepackage{amssymb,amsmath,color,graphicx}
\parindent=0pt
\parskip=16pt

\begin{document}
\centerline{Matlab Lab 1: Iterates and Cobwebs }
\centerline{\it Math 112 - Spring, 2010}
\vskip .2in
STARTING UP MATLABAND SETTING THE PATHS
\begin{enumerate}
\item Turn on machine if necessary and log on.  
\item	Download zipped Matlab file from the Sakai site.   Unzip these files into a folder somewhere in your workspace, named something that you�ll identify as related to this course. (For example: ``TotallyCoolMatlabStuff").
\item	Start Matlab (under �Programs� in the Start menu).  At the upper left of the command window you�ll see three dots in the pull-down menu.  Select this, and then select the folder you just made as your Current Directory.
\item	Basics (skip if you know this - if you don't, download ``Matlab Basics" and work through.):  
\begin{enumerate}
\item how to declare and manipulate vectors and matrices, 
\item	plotting a function of one variable, (zooming in, ginput, hold on, line styles, adding a title and labeling axes)
\item	plotting a function of two variables
\item	writing a function as an M-file and using it
\end{enumerate}
\item	Try these routines: 
\begin{enumerate}
\item 	Y=iterates(@fcn, x0, N)
\item 
cobweb(@fcn,x0,N)
\end{enumerate}
First,  set the global variable, a, in the Command Window by typing �global a� and then �a=.5�, or whatever value you want it to be.  
You can use �logistic� as the function, i.e. you type:
\begin{verbatim}
global a
a = 3.2
 Y=iterate(@logistic,.1,20);
plot(Y)
\end{verbatim}
and you should get a graph with 20 iterates of the function $f(x) = 3.2 x ( 1 - x) $ starting at $x_0 = .1$.\\
Or type:
\begin{verbatim}
a = 2;
cobweb(@logistic,.2,10)
\end{verbatim}
and you should get a cobweb graph with 10 iterates for the function $f(x) = 2 x (1 -x)$ starting at $x_0 = .2$
\item Try the built-in routine �fzero�: z=fzero(@fcn,x0), where x0 is an initial guess for the zero.  You might want to use it to find the fixed point of the equation $f(x) = \frac{1}{2} ( x + \frac{5}{x} ) $, (which should be the equal to $\sqrt{5}$, as we saw last time).\\
\begin{verbatim}
z = fzero('.5 * (x + 5/x)- x',2)
\end{verbatim}
tells Matlab to find the zero of the function $.5 ( x + \frac{5}{x})  - x$, i.e. the point where $f(x) = x$.  It also tells Matlab to use the initial guess, 2.

Try starting at  different initial guess: what happens?  Why?

Instead of actually typing in the function you can put the name of a Matlab file that returns the value of the function.  An example of this can be found in the file: FindZeroExample which finds the zero of the function stored in the file FindZeroFunction.  Check these out and see if you can modify them to answer the following:
\item	Use the built-in function fzero to find a point of period 2 for the logistic map with $a = 3.2$.   Use iterate to see if this period 2 orbit is {\it stable}.\\
{\it Hint: you first might want to write an M-file that returns $F^2(x) - x $, where $F$ is the logistic map: $F(x) = 3.2 x (1 -x)$. }
 \end{enumerate}
\end{document}